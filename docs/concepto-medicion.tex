% Options for packages loaded elsewhere
\PassOptionsToPackage{unicode}{hyperref}
\PassOptionsToPackage{hyphens}{url}
\PassOptionsToPackage{dvipsnames,svgnames*,x11names*}{xcolor}
%
\documentclass[
  12pt,
]{book}
\usepackage{lmodern}
\usepackage{setspace}
\usepackage{amsmath}
\usepackage{ifxetex,ifluatex}
\ifnum 0\ifxetex 1\fi\ifluatex 1\fi=0 % if pdftex
  \usepackage[T1]{fontenc}
  \usepackage[utf8]{inputenc}
  \usepackage{textcomp} % provide euro and other symbols
  \usepackage{amssymb}
\else % if luatex or xetex
  \usepackage{unicode-math}
  \defaultfontfeatures{Scale=MatchLowercase}
  \defaultfontfeatures[\rmfamily]{Ligatures=TeX,Scale=1}
\fi
% Use upquote if available, for straight quotes in verbatim environments
\IfFileExists{upquote.sty}{\usepackage{upquote}}{}
\IfFileExists{microtype.sty}{% use microtype if available
  \usepackage[]{microtype}
  \UseMicrotypeSet[protrusion]{basicmath} % disable protrusion for tt fonts
}{}
\makeatletter
\@ifundefined{KOMAClassName}{% if non-KOMA class
  \IfFileExists{parskip.sty}{%
    \usepackage{parskip}
  }{% else
    \setlength{\parindent}{0pt}
    \setlength{\parskip}{6pt plus 2pt minus 1pt}}
}{% if KOMA class
  \KOMAoptions{parskip=half}}
\makeatother
\usepackage{xcolor}
\IfFileExists{xurl.sty}{\usepackage{xurl}}{} % add URL line breaks if available
\IfFileExists{bookmark.sty}{\usepackage{bookmark}}{\usepackage{hyperref}}
\hypersetup{
  pdftitle={Conceptos y medición de cohesión social en proyectos internacionales},
  pdfauthor={Juan Carlos Castillo, Francisco Olivos \& Julio Iturra},
  colorlinks=true,
  linkcolor=blue,
  filecolor=Maroon,
  citecolor=Blue,
  urlcolor=Blue,
  pdfcreator={LaTeX via pandoc}}
\urlstyle{same} % disable monospaced font for URLs
\usepackage[left=4cm, right=3cm, top=2.5cm, bottom=2.5cm]{geometry}
\usepackage{longtable,booktabs}
% Correct order of tables after \paragraph or \subparagraph
\usepackage{etoolbox}
\makeatletter
\patchcmd\longtable{\par}{\if@noskipsec\mbox{}\fi\par}{}{}
\makeatother
% Allow footnotes in longtable head/foot
\IfFileExists{footnotehyper.sty}{\usepackage{footnotehyper}}{\usepackage{footnote}}
\makesavenoteenv{longtable}
\usepackage{graphicx}
\makeatletter
\def\maxwidth{\ifdim\Gin@nat@width>\linewidth\linewidth\else\Gin@nat@width\fi}
\def\maxheight{\ifdim\Gin@nat@height>\textheight\textheight\else\Gin@nat@height\fi}
\makeatother
% Scale images if necessary, so that they will not overflow the page
% margins by default, and it is still possible to overwrite the defaults
% using explicit options in \includegraphics[width, height, ...]{}
\setkeys{Gin}{width=\maxwidth,height=\maxheight,keepaspectratio}
% Set default figure placement to htbp
\makeatletter
\def\fps@figure{htbp}
\makeatother
\setlength{\emergencystretch}{3em} % prevent overfull lines
\providecommand{\tightlist}{%
  \setlength{\itemsep}{0pt}\setlength{\parskip}{0pt}}
\setcounter{secnumdepth}{5}
\usepackage[utf8]{inputenc}
\usepackage[spanish,es-tabla]{babel}
\usepackage[fixlanguage]{babelbib}
\usepackage{geometry}
\geometry{letterpaper,left=2cm,top=2cm, right=2cm}
\usepackage{times}           
\usepackage{caption}
\captionsetup[figure, table]{labelfont={bf},labelformat={default},labelsep=period}
\usepackage{graphicx}
\usepackage{float}
\usepackage{booktabs}
\usepackage{longtable}
\usepackage{array}
\usepackage{multirow}
\usepackage{wrapfig}
\usepackage{float}
\usepackage{colortbl}
\usepackage{xcolor}
\usepackage{pdflscape}
\usepackage{tabu}
\usepackage{threeparttable}
\usepackage{pdfpages} %para pdf portada

% fuente: https://stackoverflow.com/questions/45963505/coverpage-and-copyright-notice-before-title-in-r-bookdown
%\let\oldmaketitle\maketitle 
%\AtBeginDocument{\let\maketitle\relax}

% \renewcommand{\tablename}{Tabla}
% \ifxetex
%   \usepackage{polyglossia}
%   \setmainlanguage{spanish}
%   % Tabla en lugar de cuadro
%   \gappto\captionsspanish{\renewcommand{\tablename}{Tabla}
%           \renewcommand{\listtablename}{Índice de tablas}}
% \else
%   % \usepackage[spanish,es-tabla]{babel}
% \fi

\usepackage{booktabs}
\usepackage{longtable}
\usepackage{array}
\usepackage{multirow}
\usepackage{wrapfig}
\usepackage{float}
\usepackage{colortbl}
\usepackage{pdflscape}
\usepackage{tabu}
\usepackage{threeparttable}
\usepackage{threeparttablex}
\usepackage[normalem]{ulem}
\usepackage{makecell}
\usepackage{xcolor}
\ifluatex
  \usepackage{selnolig}  % disable illegal ligatures
\fi
\newlength{\cslhangindent}
\setlength{\cslhangindent}{1.5em}
\newlength{\csllabelwidth}
\setlength{\csllabelwidth}{3em}
\newenvironment{CSLReferences}[3] % #1 hanging-ident, #2 entry spacing
 {% don't indent paragraphs
  \setlength{\parindent}{0pt}
  % turn on hanging indent if param 1 is 1
  \ifodd #1 \everypar{\setlength{\hangindent}{\cslhangindent}}\ignorespaces\fi
  % set entry spacing
  \ifnum #2 > 0
  \setlength{\parskip}{#2\baselineskip}
  \fi
 }%
 {}
\usepackage{calc} % for \widthof, \maxof
\newcommand{\CSLBlock}[1]{#1\hfill\break}
\newcommand{\CSLLeftMargin}[1]{\parbox[t]{\maxof{\widthof{#1}}{\csllabelwidth}}{#1}}
\newcommand{\CSLRightInline}[1]{\parbox[t]{\linewidth}{#1}}
\newcommand{\CSLIndent}[1]{\hspace{\cslhangindent}#1}

\title{Conceptos y medición de cohesión social en proyectos internacionales}
\author{Juan Carlos Castillo, Francisco Olivos \& Julio Iturra}
\date{2021-03-25}

\begin{document}
\maketitle

{
\hypersetup{linkcolor=}
\setcounter{tocdepth}{1}
\tableofcontents
}
\listoftables
\listoffigures
\setstretch{1.5}
\hypertarget{introducciuxf3n}{%
\chapter{Introducción}\label{introducciuxf3n}}

Este documento tiene por objetivo describir y resumir diferentes aproximaciones conceptuales y metodológicas hacia el concepto de cohesión social. Originalmente fue desarrollado para servir como insumo en la generación del \href{https://ocs-coes.netlify.app/}{Observatorio de Cohesión Social (OCS)} del Centro de Estudios de Conflicto y Cohesión Social (COES), cuyo objetivo es proponer indicadores de cohesión social con foco en américa latina basados en datos secundarios y permitir la realización de análisis en una plataforma interactiva.

Como punto de partida de este esfuerzo por definir y operacionalizar la cohesión social para el OCS es necesario considerar algunos trabajos de carácter teórico que existen en América Latina. Uno de estos casos es el trabajo realizado tempranamente por CEPAL (\protect\hyperlink{ref-ottone2007cohesion}{Ottone et al., 2007}) en respuesta a una serie de problemas que afectan al continente, como son los altos índices de pobreza, desigualdad y diversas formas de discriminación y exclusión. La cohesión social en este caso está estrechamente vinculada a los valores democráticos y el respeto al estado de derecho. Desde esta perspectiva, la cohesión social tendría dos vertientes principales: por un lado referiría a la capacidad de los mecanismos instituidos de inclusión social (sistema educacional, protección social, titularidad de derecho, etc.), y por otro lado a los comportamientos y valores de los sujetos que conforman la sociedad (sentido de pertenencia, capital social, confianza en las instituciones, solidaridad, etc.)

Una aproximación conceptual complementaria a la de CEPAL a finales de la década del 2010 proviene del trabajo de CIEPLAN desarrollado en torno al proyecto y encuesta ECOsociAL (2007) (\protect\hyperlink{ref-tironibarrios_Redes_2008}{Tironi Barrios \& Foxley Ríoseco, 2008}). Desde esta perspectiva se señala que el concepto de cohesión social es tanto descriptivo como normativo, ya que no solo permite describir los elementos que unen a la sociedad, sino también propone ciertos ideales a alcanzar como sociedad en contextos democráticos. Así, una sociedad cohesionada no sería una sociedad cerrada en torno a determinados valores, sino que permitiría que los sujetos se relacionen en torno a principios de justicia que dan fundamento al actuar cooperativo. Por eso, desde esta perspectiva el concepto supone un sentido de pertenencia moral y sujeción a reglas al estilo durkhemiano, que a su vez son compatibilizados con la autonomía individual para las sociedades modernas. Sin embargo, la conciencia moral podría incluir elementos como la misma autonomía individual, la diferencia y la diversidad como principios orientadores.

El auge observado por el concepto de cohesión social a finales de la década del 2010 en Chile y América Latina, impulsado principalmente por CEPAL, PNUD y CIEPLAN, entra luego a una fase de menor actividad en la región, lo que contrasta con el desarrollo de proyectos a larga escala a nivel internacional enfocados principalmente en la medición de cohesión social. El presente documento pone su foco en esta segunda fase, presentando una revisión de los proyectos más emblemáticos a nivel internacional que han puesto el foco en medición de cohesión social.

La revisión de experiencias internacionales en el monitoreo de cohesión social tiene cuatro focos centrales. En primer lugar, conocer en mayor detalle la estructura organizacional y objetivos de los proyectos existentes actualmente. En segundo lugar, revisar las definiciones conceptuales de cohesión social para entregar un marco general que permita especificar dimensiones y subdimensiones del concepto. El tercer foco es la operacionalización que estas experiencias realizan del concepto para identificar o generar indicadores. Finalmente, cada revisión de las experiencias internacionales se concluye con un listado de referencias comentadas en donde se puede encontrar documentos de trabajo o aplicaciones de los datos levantados por cada uno de los proyectos.

La selección de iniciativas internacionales buscó ser representativa geográficamente, dado que las perspectivas sobre la cohesión social poseen un arraigo en los atributos socioculturales en los que estos se desarrollan: vínculos sociales o capital social en Estados Unidos por ejemplo o ampliación del Estado de Bienestar a grupos excluidos en Europa. Así, nuestra selección incluye el Scalon-Monash Index of Social Cohesión de Australia, el Social Cohesión Radar alemán con alcance internacional, el trabajo teórico Civic Engagement and Social Cohesion Report de los Estados Unidos, y el clásico trabajo de Jenson (\protect\hyperlink{ref-jenson1998mapping}{1998}, \protect\hyperlink{ref-jenson2010defining}{2010}) Mapping Social Cohesion en Canadá. Finalmente, y si bien no calza totalmente con el carácter y momento de los otros estudios, se incluyó también al proyecto ECOsociAL que comprendió a siete ciudades de Latinoamérica.

\hypertarget{mapping-social-cohesion}{%
\chapter{Mapping Social Cohesion}\label{mapping-social-cohesion}}

\hypertarget{descripciuxf3n-general}{%
\section{Descripción General}\label{descripciuxf3n-general}}

El trabajo de \protect\hyperlink{ref-jenson1998mapping}{Jenson} (\protect\hyperlink{ref-jenson1998mapping}{1998}) en Canadá es uno de los trabajos
pioneros que trató de dar orden a las discusiones sobre cohesión social.
Se reconoce que a fines del siglo XX los organismos gubernamentales y no
gubernamentales tenían una gran preocupación por el tema, pero no
existía una definición clara de lo que implicaba ni tampoco una diferenciación de otros conceptos asociados.

El trabajo de Jenson nace de la Mesa de trabajo del Canadian Policy
Research Network ``Mapping Social Cohesion'' de diciembre de 1997. El
objetivo fue identificar la literatura emergente en el tema y definir
una agenda de investigación para poder operacionalizar las aproximaciones conceptuales que caracterizaban los estudios de cohesión social. El estudio revisa
analíticamente los orígenes del debate en la tradición sociológica hasta
los conceptos más actuales, incluyendo el estado del arte en Canadá.
Jenson señala textualmente que la ``clarificación es el objetivo primario
de este paper'' (1998: 2).

El trabajo fue financiado por Canadian Heritage, el Department of
Justice Canada y The Kahanoff Foundation Nonprofit Sector Research
Initiative.

\hypertarget{concepto-de-cohesiuxf3n-social}{%
\section{Concepto de Cohesión Social}\label{concepto-de-cohesiuxf3n-social}}

La conceptualización de \protect\hyperlink{ref-jenson1998mapping}{Jenson} (\protect\hyperlink{ref-jenson1998mapping}{1998}) ha sido un recurso de suma importancia para trabajos posteriores como el Scalon-Monash Index australiano, que lo utiliza como punto de partida. Para Jenson (1998), la cohesión social es más un proceso que un estado o condición, involucrando un sentido de compromiso, y deseo o capacidad de vivir juntos en armonía. La revisión conceptual lleva a plantear que la literatura ha reconocido a la cohesión social como una característica de la
sociedad y no de los individuos.

El concepto de cohesión social es un concepto con historia, y partiendo de esta premisa es que \protect\hyperlink{ref-jenson2010defining}{Jenson} (\protect\hyperlink{ref-jenson2010defining}{2010}) plantea que han existido tres familias vinculadas al concepto con diferentes aplicaciones empíricas:

\begin{itemize}
\item
  La primera vincula la cohesión social a la noción de \emph{inclusión social}. A comienzo de los '80 la OCDE decidió revisar el concepto de cohesión social en orden a hacerlo compatible con la reestructuración económica de los países miembros y a la vez sustentable en el tiempo. En este uso, se desprendía que la OCDE estaba homologando cohesión social con estabilidad social. Ya a comienzos del siglo XXI la UE orientaba sus políticas sociales en la misma dirección y definía el desarrollo económico y la cohesión social como objetivos principales. El Council of Europe (organismo de la Unión Europea) señalaba explícitamente: \emph{"Social cohesion, as defined by the Directorate General of Social Cohesion of the Council of Europe, is a concept that includes values and principles which aim to ensure that all citizens, without discrimination and on an equal footing, have access to fundamental social and economic rights}``. Se debe considerar que en esta conceptualización se tomó cuidado en sugerir que la integración no implicaba necesariamente formas tradicionales de integración social, siendo un concepto para una sociedad abierta y multicultural. No obstante, \protect\hyperlink{ref-jenson2010defining}{Jenson} (\protect\hyperlink{ref-jenson2010defining}{2010}) agrega que realmente no se estaba entregando una conceptualización clara de lo que significaba cohesión social:''it does not define social cohesion as such but seeks to identify some of the factors in social cohesion \ldots" (Jenson, 2010:5)
\item
  La segunda línea de conceptualización de la cohesión social lo hace en relación a \emph{capital social} o, incluso, como sinónimos. Para esta tradición el capital social es la herramienta práctica para lograr la cohesión social. El capital social no solo entrega recursos a los individuos, como acceso a servicios especiales a menor costo o para que alguien cuide al hijo de una madre que debe ir a trabajar, sino que produce un bien colectivo al tener sociedades más saludables. A esta conceptualización liderada por Putnam, le siguieron una serie de críticas sobre su carácter positivo y si realmente era social considerando su enfoque focalizado en el individuo. Esto produjo el refinamiento del concepto, incorporándose los conceptos de la teoría del capital, tales como el \emph{social bridging}, \emph{bonding} y \emph{linking}.
\item
  Y finalmente, una tercera familia de estudios se focaliza en las instituciones y la gobernanza liderada principalmente por el Banco Mundial. Esta definición se superpone a las dos anteriores, pero toma un carácter distinto. Señalan que los resultados económicos dependen de instituciones efectivas, y esta a su vez de la cohesión social. Por lo tanto, construir cohesión social es una tarea vital para el desarrollo. El desafío está puesto en la inclusión de las personas en la provisión de servicios públicos de forma equitativa y eficiente.
\end{itemize}

Si bien \protect\hyperlink{ref-jenson2010defining}{Jenson} (\protect\hyperlink{ref-jenson2010defining}{2010}) no entrega una definición propia de la cohesión social, sino que una revisión del estado del arte en la materia, la Social Cohesion Network reemplaza su definición de cohesión social basada en valores por una más funcionalista fijada en las conductas: ``la cohesión social está basada en la disposición de los individuos a cooperar y trabajar en todos los niveles de la sociedad para lograr metas comunes'' (\protect\hyperlink{ref-Jeannote2003}{Jeannotte, 2003}).

La ventaja de esta definición, y que la ha llevado a ser un referente para posteriores experiencias en todo el mundo, es que se focaliza claramente en los resultados de la cohesión social y en el conjunto de indicadores para medir el stock de cohesión social en la sociedad (trabajar juntos, metas colectivas, cooperación, etc). La tarea se simplifica si consideramos desde la definición la igualdad de oportunidades, las metas compartidas y valores.

\newpage

\hypertarget{operacionalizaciuxf3n}{%
\section{Operacionalización}\label{operacionalizaciuxf3n}}

\begin{figure}[H]

{\centering \includegraphics[width=0.75\linewidth]{inputs/images/mapping} 

}

\caption{Operacionalización de Mapping Social Cohesion}\label{fig:mapping}
\end{figure}

El concepto de cohesión social es operacionalizado inicialmente en 5 dimensiones.

\begin{enumerate}
\def\labelenumi{\arabic{enumi}.}
\item
  Pertenencia/aislamiento: esta dimensión es más o menos compartida por toda la literatura, y señala el grado en que los sujetos comparten valores e identidades colectivas. Valores compartidos permiten desarrollar un sentido de pertenencia y a la vez generar un sentido de comunidad.
\item
  Inclusión/exclusión: esta dimensión tiene un carácter institucional y en el mercado en particular como elemento central de la modernidad.
\item
  Participación/no-involucramiento: la cohesión social requiere involucramiento de parte de los sujetos. El desencantamiento con la política pondría en riesgo la cohesión social y obliga también a los gobiernos a fortalecer el tercer sector.
\item
  Reconocimiento/rechazo: se parte de la idea de que las sociedades modernas son pluralistas en su sistema de valores. El pluralismo es un bien y la tolerancia una meta. El rechazo, la intolerancia y excesivos esfuerzos por lograr la unanimidad irían contra la idea de reconocimiento de la diversidad.
\item
  Legitimidad/ilegitimidad: se señala que la intermediación necesaria para vivir con los conflictos de valores no depende de los sujetos, sino que es producto de las instituciones. Por lo tanto, en última instancia la cohesión social depende de la legitimidad que tienen las instituciones públicas y privadas en asegurar la cohesión social.
\end{enumerate}

Con esto en mente, \protect\hyperlink{ref-jenson2010defining}{Jenson} (\protect\hyperlink{ref-jenson2010defining}{2010}) plantea tres set de indicadores. Un primer set de indicadores (1 - 5) corresponde a mediciones de disparidades sociales, y las brechas indican acceso desigual a recursos económicos y a servicios sociales básicos.

\begin{enumerate}
\def\labelenumi{\arabic{enumi}.}
\tightlist
\item
  Cohesión social como inclusión social relacionada al acceso de recursos financieros:
\end{enumerate}

\begin{itemize}
\tightlist
\item
  Coeficiente de Gini
\item
  Mediciones de ingresos
\item
  Mediciones de pobreza (población viviendo con menos de 1 dolar, población viviendo con menos de 2 dólares y población bajo la línea de pobreza).
\end{itemize}

\begin{enumerate}
\def\labelenumi{\arabic{enumi}.}
\setcounter{enumi}{1}
\tightlist
\item
  Cohesión social como inclusión social relacionada la actividad económica:
\end{enumerate}

\begin{itemize}
\tightlist
\item
  Tasa de empleo (juvenil, femenina, minorías e inmigrantes)
\item
  Empleo en economía informal, como porcentaje del empleo total
\end{itemize}

\begin{enumerate}
\def\labelenumi{\arabic{enumi}.}
\setcounter{enumi}{2}
\tightlist
\item
  Cohesión social como inclusión social relacionada con el acceso a educación y capital humano:
\end{enumerate}

\begin{itemize}
\tightlist
\item
  Tasa de alfabetización (total, hombres y mujeres)
\item
  Porcentaje de población sobre 15 años que no ha completado educación primaria (total, hombres y mujeres)
\item
  Porcentaje de población sobre 20 años que no ha completado educación secundaria (total, hombres y mujeres)
\item
  Porcentaje de niños en edad de educación secundaria matriculados en educación secundaria
\item
  Porcentaje de población entre 18 y 24 años en educación terciaria
\end{itemize}

\begin{enumerate}
\def\labelenumi{\arabic{enumi}.}
\setcounter{enumi}{3}
\tightlist
\item
  Cohesión social como inclusión social relacionada con el acceso a la salud:
\end{enumerate}

\begin{itemize}
\tightlist
\item
  Esperanza de vida al nacer (en años)
\item
  Mortalidad infantil (por mil nacidos vivimos)
\item
  Mortalidad bajo 5 años (por mil)
\item
  Nacidos atendidos por equipo medido (porcentaje del total para toda la población y para las minorías)
\end{itemize}

\begin{enumerate}
\def\labelenumi{\arabic{enumi}.}
\setcounter{enumi}{4}
\tightlist
\item
  Cohesión social como inclusión social relacionada con el acceso a tecnología:
\end{enumerate}

\begin{itemize}
\tightlist
\item
  Porcentaje de hogares con internet de banda ancha
\end{itemize}

El segundo tipo de indicadores corresponden a medidas de homogeneidad cultural y étnicas vinculadas a la dimensión de diversidad de la cohesión social. Más diversidad es considerado para Jenson (2010) como un indicador de menos cohesión social.

\begin{enumerate}
\def\labelenumi{\arabic{enumi}.}
\setcounter{enumi}{5}
\tightlist
\item
  Cohesión social como homogeneidad cultural y étnica
\end{enumerate}

\begin{itemize}
\tightlist
\item
  Porcentaje de extranjeros en la población
\item
  Fraccionalización étnica - índice que mide la probabilidad de que dos personas seleccionadas aleatoriamente pertenezcan al mismo grupo etnolinguístico
\end{itemize}

El tercer y último grupo de indicadores (7 - 8) corresponde a las dimensiones de pertenencia y participación.

\begin{enumerate}
\def\labelenumi{\arabic{enumi}.}
\setcounter{enumi}{6}
\tightlist
\item
  Cohesión social como confianza
\end{enumerate}

\begin{itemize}
\tightlist
\item
  Preguntas sobre confianza desde encuestas de opinión pública. El autor plantea explícitamente la encuesta mundial de valores
\end{itemize}

\begin{enumerate}
\def\labelenumi{\arabic{enumi}.}
\setcounter{enumi}{7}
\tightlist
\item
  Cohesión social como participación y solidaridad
\end{enumerate}

\begin{itemize}
\tightlist
\item
  Participación electoral como porcentaje de electores posibles que participan en elecciones nacionales
\item
  Tasa de participación en asociaciones voluntarias. Nuevamente se hace referencia a la encuesta mundial de valores.
\item
  Caridad. Porcentaje de personas que realizan donaciones.
\end{itemize}

\hypertarget{referencias}{%
\section{Referencias}\label{referencias}}

\begin{enumerate}
\def\labelenumi{\arabic{enumi}.}
\item
  Jeannotte, M. (2003). Social Cohesion: Insights From Canadian
  Research. Presented at the Conference on Social Cohesion, Hong Kong.

  En esta presentación la autora resume el trabajo de Jenson y la influencia que este ha tenido en la conceptualización de la cohesión social. Asimismo, es importante destacar que es complementado en base a los trabajos de Bernard en Canadá.
\item
  Jenson, J. (1998). Mapping social cohesion: The state of Canadian
  research (No.~F03). Ottawa: CPRN.

  Es el trabajo principal de desarrollo teórico de Jenson. Con este informe el concretiza el encargo de dar un panorama general de las conceptualizaciones de cohesión social
\item
  Jenson, J. (2010). Defining and Measuring Social Cohesion. Londres:
  Commonwealth Secretariat.

  En este libro, el autor actualiza la conceptualización de cohesión social y desarrolla el conjunto de indicadores para medirla que han sido incluidos en este informe
\item
  Jenson, J., \& Saint-Martin, D. (2003). New routes to social cohesion? Citizenship and the social investment state. Canadian
  Journal of Sociology, 28(1), 77--99.

  Este artículo es un ejemplo de la aplicación del trabajo que ha desarrollado Jenson desde la Universidad de Montreal como investigadora. Aquí plantea la importancia de la consideración de la cohesión para el rediseño de la estructura de bienestar y de la inversión social que deben realizar las comunidades políticas.
\end{enumerate}

\hypertarget{scanlon-monash-index-of-social-cohesion}{%
\chapter{Scanlon-Monash Index of Social Cohesion}\label{scanlon-monash-index-of-social-cohesion}}

\hypertarget{descripciuxf3n-general-1}{%
\section{Descripción General}\label{descripciuxf3n-general-1}}

Este estudio ha sido realizado en Australia anualmente desde el año 2007 (con excepción del año 2008), siendo a la fecha el último estudio disponible el año 2019. La Fundación Scalon, responsable del estudio, tiene como fin desde
su creación en el 2001 contribuir a que Australia sea un país acogedor,
próspero y cohesivo.

En este marco, el estudio tiene por objetivo saber si en el futuro puede ser sostenible la migración y la cohesión social que han caracterizado a Australia desde la Segunda Guerra Mundial. De esta forma, el Monash Institute for the Study of Global Movements y el Australian Multicultural Foundation, con fondos de la Fundación Scanlon, encargó al profesor Dr.~Andrew Markus (\protect\hyperlink{ref-markus2013mapping}{A. Markus, 2014}; \protect\hyperlink{ref-markus_Attitudinal_2007}{A. B. Markus \& Arunachalam, 2007}) de la Monash University diseñar y dirigir una medición de la cohesión social que sería repetida cada dos años. El estudio fue conducido por el Melbourne-based Social Research Centre, y ha sido realizado desde el 2013 con fondos del gobierno federal, lo que demuestra un trabajo que ha involucrado a organizaciones sin fines de lucro, academia y ahora al gobierno.

La cohesión social no es medida en abstracto, sino que se examina en el contexto del impacto social de un periodo prolongado de inmigración significativa y sostenida. En los últimos años se ha profundizado con encuestas comparadas en áreas de alto nivel de inmigración. Esto es de relevancia ya que uno de los hallazgos más importantes del estudio ha sido que la inmigración es un recurso que genera tensión social.

El proyecto tiene como objetivo explícito estimular a la discusión basada en evidencia sobre el crecimiento de la población australiana y la relación entre inmigración y cohesión social. Para esto, el componente principal es la disposición de un sitio web con la información de los estudios (\href{https://scanlonfoundation.org.au/mapping-social-cohesion/}{Mapping Social Cohesión}). El sitio entrega información para la discusión de políticas públicas sobre inmigración y cohesión social en base a los datos de la encuesta.

En cuando a datos de diseño y muestra, el sitio web del Scanlon-Monash los resume de la siguiente manera para el último reporte del año 2019:

\begin{figure}[H]

{\centering \includegraphics[width=0.75\linewidth]{inputs/images/scanlon2019} 

}

\caption{Operacionalización del Scanlon-Monash Index of Social Cohesion}\label{fig:mapping2}
\end{figure}

\hypertarget{concepto-de-cohesiuxf3n-social-1}{%
\section{Concepto de Cohesión Social}\label{concepto-de-cohesiuxf3n-social-1}}

El concepto que utilizan en el SMISC (Scanlon-Monash Index of Social Cohesion) recoge una visión ecléctica de la tradición académica. Su conceptualización ha estado principalmente influida por los trabajos de los canadienses Paul Bernard y Jane Jenson, quienes señalan que el objetivo general de la cohesión social es que todos los ciudadanos puedan acceder en iguales condiciones a los derechos económicos y sociales fundamentales.

En el desarrollo del proyecto se reconoce la tradición que destaca la importancia del papel que juegan el consenso y el conflicto en las sociedades. El interés en este concepto ha sido alentado por los procesos sociales generales como la globalización, el cambio económico y la guerra contra el terrorismo en las últimas décadas. Sin embargo, la creación de este índice se enfrenta al problema más común en la conceptualización de teorizaciones acerca de la social: no hay acuerdo en la definición del concepto.

Según los autores del índice, las definiciones actuales muchas veces son intangibles, y refieren a una característica de un grupo o la disposición de lo sujetos para participar y compartir objetivos. En este marco, y en el intento de concretizar la conceptualización con un enfoque ecléctico se llegó a la identificación de tres elementos comunes: la cohesión social es una visión compartida, una propiedad grupal y un proceso.

La \emph{visión compartida} refiere a que los investigadores reconocen que la cohesión social requiere de valores universales, respeto mutuo y aspiraciones comunes o identidades compartidas por los miembros. Asimismo, hay acuerdo en que este constructo es una \emph{propiedad de los grupos o comunidades} en los que existe esa visión compartida sobre metas y responsabilidades. Finalmente, la cohesión social es comúnmente vista como \emph{un continuo y un proceso} sin fin de búsqueda de la armonía social y no un resultado como se podría pensar.

Por otro lado, la diferencia en las definiciones de cohesión social están puestas fuertemente en los factores que impactan sobre el proceso de armonía de los grupos o comunidades. Esto último debe ser considerado por el OCS para identificar aquellas variables que permitirán segmentar los distintos indicadores de cohesión social buscando explicar la variabilidad entre grupos y países. Los factores claves que reconocen en el SMISC son económicos, políticos y socioculturales. En cuanto a los factores económicos, se reconocen los niveles de desempleo y pobreza, distribución del ingreso, migraciones, salud, satisfacción con la vida y sentido de seguridad, y la responsabilidad del gobierno por la pobreza y la desigualdad. En lo político, los factores que influyen sobre la cohesión social están relacionados al nivel de participación política e involucramiento social, incluyendo el voluntariado, el desarrollo de capital social, y las normas y confianza social que facilitan la cooperación en los grupos o comunidades. Finalmente, a nivel sociocultural se considera el consenso y la divergencia en torno a temas de significación nacional y local.

En resumen, la definición que el SMISC entrega sobre cohesión social es la disposición que tienen los miembros de una sociedad para cooperar con cada uno de los demás para la sobrevivencia y la prosperidad. En virtud de esto, el SMISC cubre cinco dominios de la cohesión social:

\begin{enumerate}
\def\labelenumi{\arabic{enumi}.}
\item
  Pertenencia: esta dimensión comprende los valores que comparte la población, la identificación con el país (Australia) y la confianza.
\item
  Valoración: comprende la satisfacción con la situación económica, satisfacción con la vida, felicidad y expectativas sobre el futuro.
\item
  Justicia social y equidad: visión sobre el adecuado apoyo a las personas de bajos ingresos, la brecha entre ricos y pobres, el país como una tierra de oportunidades y confianza en el gobierno.
\item
  Participación (política): trabajo voluntario, actividad e involucramiento político.
\item
  Aceptación, rechazo y legitimidad: experiencia de discriminación, actitudes hacia las minorías y \emph{newcomers}.
\end{enumerate}

\hypertarget{operacionalizaciuxf3n-1}{%
\section{Operacionalización}\label{operacionalizaciuxf3n-1}}

Uno de los elementos característicos de la iniciativa australiana es el diseño de un índice de cohesión social, el cual ha sido medido en cada una de las versiones del estudio. Así, el \textbf{Scanlon-Monash Index de Cohesión Socia}l esta compuesto por indicadores en cada una de las cinco dimensiones mencionadas previamente.

\begin{figure}[H]

{\centering \includegraphics[width=0.75\linewidth]{inputs/images/scalon} 

}

\caption{Operacionalización del Índice de Cohesión Social}\label{fig:scalon}
\end{figure}

La dimensión de \textbf{sentido de pertenencia} incluye preguntas generales sobre pertenencia, orgullo y la importancia de mantener la cultura y forma de vida de los australianos. Los dos primeros medidos en una escala de cuatro puntos desde ``To a great extent'' a ``Not at all.'' La importancia de la cultura y forma de vida utiliza una escala de acuerdo de 4 puntos.

La segunda dimensión de \textbf{valoración} se compone por un indicador de satisfacción con la situación financiera y felicidad en los últimos años.

La \textbf{equidad y justicia social} se mide con con el nivel de acuerdo acerca de que Australia es una tierra de oportunidades en donde el trabajo duro entrega una mejor vida; el grado de acuerdo con que existe una gran diferencia entre altos ingresos y bajos ingresos; que las personas con bajos ingresos reciben suficiente apoyo; y que el gobierno federal hace las cosas correctas por las personas.

La dimensión de \textbf{participación política} es medida a través de un índice que combina distintas formas de participación: votar en elecciones, firmar una petición, contacto con miembros del parlamento, unirse a un boycot de algún producto comercial y asistir a una protesta, marcha o demostración.

Finalmente, la dimensión de \textbf{aceptación y rechazo} se compone de indicadores de experiencia de discriminación en base a color de piel, etnia o religión; pesimismo sobre el futuro preguntado a través de si cree que en 3 años su vida en Australia estará peor, igual o mejor; si cree que el gobierno australiano debe dar asistencia para que las minorías étnicas mantengan sus costumbres y tradiciones; y si cree que aceptar inmigrantes de muchos países hace a Australia más fuerte. (\protect\hyperlink{ref-colic-peisker_Social_2015}{Colic-Peisker \& Robertson, 2015})

\hypertarget{referencias-1}{%
\section{Referencias}\label{referencias-1}}

\begin{enumerate}
\def\labelenumi{\arabic{enumi}.}
\item
  Markus, A. \& Dharmalingam, A. (2014). Mapping Social Cohesion: The
  Scanlon Foundation surveys 2014.

  Este informe presenta los resultados de la encuesta de Scalon Foundation. En cada informe se resume la construcción del índice y las dimensiones que comprenden en su propuesta de operacionalización de la cohesión social.
\item
  Colic-Peisker, V. \& Robertson, S. (2015). Social change and
  community cohesion: an ethnographic study of two Melbourne suburbs.
  Ethnic and Racial Studies, 38(1), 75-91.

  Dado el importante énfasis local que ha tenido esta encuesta, ha
  generado problematizaciones que han sido cubiertas por otros
  investigadores como es el ejemplo de este estudio etnográfico. Los
  autores se plantean la pregunta de hasta qué punto comunidades
  etnoculturalmente diversas pueden ser cohesivas.
\item
  Markus, A., \& Dharmalingam, A. (2007). Attitudinal Divergence in a
  Melbourne Region of High Immigrant Concentration: A Case Study.
  People and Place, 15(4), 38-48.

  En este artículos los autores utilizan datos de las encuestas
  Scalon-Monash para comprender las actitudes hacia la inmmigración y
  la diversidad étnica en dos suburbios de Melbourne. Los resultados
  sugieren que existe un apoyo hacia la no-selección en la aplicación
  de políticas sociales en base a criterios étnicos, aunque hay cierto
  rechazo a la inmigración y algunos aspectos del multiculturalismo.
\end{enumerate}

\hypertarget{social-cohesion-radar}{%
\chapter{Social Cohesion Radar}\label{social-cohesion-radar}}

\hypertarget{descripciuxf3n-general-2}{%
\section{Descripción General}\label{descripciuxf3n-general-2}}

El Social Cohesion Radar (SCR) es un esfuerzo por aportar información aldebate público sobre cohesión social, un mejor entendimiento de loscambios actuales y la proyección en el futuro. Su objetivo explícito esproveer al público general de una buena visualización de conjunto de losniveles y tendencias de la cohesión social tanto como una comprensión enprofundidad de sus determinantes y resultados (\protect\hyperlink{ref-dragolov_Social_2016}{Dragolov et al., 2016}).

El radar ha comparado la tendencia en la cohesión social de 34 países entre 1989 y 2012. La encuesta ha incluido 27 países miembros de la Unión Europea (sin Croacia) tanto como 7 naciones industrializadas de occidente (Estados Unidos, Canadá, Noruega, Suiza, Israel, Australia y Nueva Zelanda). Actualmente se trabaja en la identificación de causas, efectos y futuras tendencias. Esta es una iniciativa de Bertelsmann Stiftung en cooperación con académicos de la \href{https://www.bertelsmann-stiftung.de/en/our-projects/social-cohesion}{Jacobs University Bremen}.

Uno de los rasgos particulares del radar es que está hecho completamente en base a datos secundarios, es decir se reutilizan otros datos recopilados por los mismos u otros investigadores para medir cuestiones similares en los países objetos de estudio. Se señala que la utilización de datos secundarios tiene como ventaja que se realiza en base a datos validados y confiables de encuesta de gran escala en diferentes países. Dado que se reducen considerablemente los costos, hay un mayor alcance del estudio. Y, finalmente, es posible realizar el estudio de forma retrospectiva.

\hypertarget{concepto-de-cohesiuxf3n-social-2}{%
\section{Concepto de Cohesión Social}\label{concepto-de-cohesiuxf3n-social-2}}

La conceptualización del SCR no difiere en gran medida de la experiencia australiana, partiendo de la misma base respecto a la incapacidad de haber generado una conceptualización unívoca del concepto. No obstante sí se ha hecho cargo de la diversidad de dimensiones que la conforman.

En este proyecto, se identifican de la misma forma tres aspectos comunes a las distintas definiciones de cohesión social. Al igual que SMISC la cohesión social es un atributo del colectivo y no de los sujetos en particular. También existe una gradualidad de la cohesión social, de tal forma que las sociedades, ciudades, barrios o grupos pueden ser más o menos cohesivos. Y, finalmente, los individuos pueden experimentar distintos niveles de bienestar con distintos niveles de cohesión social. Este último aspecto es un elemento no señalado explícitamente en la definición utilizada por el SMISC y que plantearía que la cohesión social no es un concepto positivo con valor en sí mismo, sino que se debe contextualizar.

La definición que ha elaborado el equipo del radar es que ``Una sociedad cohesionada se caracteriza por relaciones sociales resilientes, conexiones emocionales positivas entre los miembros y la comunidad y una foco pronunciado en el bien común. Las relaciones sociales, en este contexto, son la red horizontal que existe entre los individuos y los grupos dentro de la sociedad. La conectividad refiere a los vínculos positivos entre individuos y sus países e instituciones. Un foco en el bien común finalmente, es reflejado en las acciones y actitudes de los miembros de la sociedad que demuestran responsabilidad por otros y por la comunidad como un todo. Estos son los tres aspectos centrales de la cohesión'' (\protect\hyperlink{ref-delhey_Happier_2016}{Delhey \& Dragolov, 2016}; \protect\hyperlink{ref-dragolov2013social}{Dragolov et al., 2013})

\hypertarget{operacionalizaciuxf3n-2}{%
\section{Operacionalización}\label{operacionalizaciuxf3n-2}}

El concepto utilizado en el Radar de Cohesión Social esta compuesto por tres dimensiones que a su vez se dividen en tres sub-dimensiones cada una. Estas dimensiones han sido evaluadas en cuatro periodos de tiempo distinto con diferentes indicadores producto de la disponibilidad de datos.

\begin{figure}[H]

{\centering \includegraphics[width=0.75\linewidth]{inputs/images/radar} 

}

\caption{Operacionalización del Radar de Cohesión Social}\label{fig:radar}
\end{figure}

\begin{enumerate}
\def\labelenumi{\arabic{enumi}.}
\tightlist
\item
  La primera dimensión de \textbf{relaciones sociales}, en tanto redes horizontales que existen entre los individuos y grupos de la sociedad, está compuestas por redes sociales, confianza en las personas y aceptación de la diversidad. Una sociedad cohesionada debería presentar fuertes redes sociales, alto nivel de confianza en otros y considerar a individuos con diferentes estilos de vida y valores como iguales.
\end{enumerate}

\textbf{Subdimensiones e indicadores:}

1.1 \emph{Redes sociales}

\begin{itemize}
\tightlist
\item
  Count on to help
\item
  How much time during past week you felt lonely
\item
  Support if needed advice on serious personal or family matter
\item
  How often socially meet with friends, relatives or colleagues
\end{itemize}

1.2 \emph{Confianza en las personas}

\begin{itemize}
\tightlist
\item
  People try to be fair
\item
  Most of the time people helpful
\item
  People can be trusted
\end{itemize}

1.3 \emph{Aceptación de la diversidad}

\begin{itemize}
\tightlist
\item
  Country's culture undermined by immigrants
\item
  Rating of religious tension (high score, low tension)
\item
  City/area good place for: Racial/ethnic minorities
\item
  Rating of ethnic tension (high score, low tension)
\item
  City/area good place for: Gay or lesbian people
\item
  Gays and lesbians free to live life as they wish
\end{itemize}

\begin{enumerate}
\def\labelenumi{\arabic{enumi}.}
\setcounter{enumi}{1}
\tightlist
\item
  La segunda dimensión de la cohesión es la \textbf{conectividad} entendida como los vínculos entre individuos, su país y sus instituciones. A su vez, esta dimensión está compuesta por tres subdimensiones: la identificación, la confianza en las instituciones y percepción de justicia. Un grupo altamente cohesionado debería tener una fuerte conexión e identificación con él, alta confianza en las instituciones y sentir que son tratados de forma justa.
\end{enumerate}

\textbf{Subdimensiones e indicadores:}

2.1 \emph{Identificación}

\begin{itemize}
\tightlist
\item
  How attached to country
\item
  Ideally, would permanently move to another country
\end{itemize}

2.2 \emph{Confianza en las instituciones}

\begin{itemize}
\tightlist
\item
  Trust in parliament
\item
  Trust in political parties
\item
  Confidence in judicial system
\item
  Confidence in local police
\item
  Honesty of elections
\item
  Confidence in health care
\item
  Confidence in financial institutions
\end{itemize}

2.3 \emph{Percepción de justicia}

\begin{itemize}
\tightlist
\item
  To get ahead need to be corrupt
\item
  Corruption within businesses
\item
  Government should reduce differences in income levels
\item
  Corruption (high score, low corruption)
\item
  Tensions between the rich and the poor
\item
  I earn what I deserve
\item
  Pay about just for me
\end{itemize}

\begin{enumerate}
\def\labelenumi{\arabic{enumi}.}
\setcounter{enumi}{2}
\tightlist
\item
  La tercera y última dimensión corresponde al \textbf{foco en el bien común}, cuyas tres subdimensiones son la solidaridad y amabilidad, respeto por las normas sociales y participación cívica. En este caso, un grupo cohesionado se caracteriza por que los miembros están preocupados por el bienestar de unos y otros, respeto y aceptación de reglas y normas, y participación en vida social y política. Los indicadores del cuarto periodo de tiempo son:
\end{enumerate}

\textbf{Subdimensiones e indicadores:}

3.1 \emph{Solidaridad y amabilidad}

\begin{itemize}
\tightlist
\item
  Helped a stranger
\item
  Unpaid voluntary work through community and social ser-vices
\item
  Donated money
\end{itemize}

3.2 \emph{Respeto por las normas sociales}

\begin{itemize}
\tightlist
\item
  How wrong to commit traffic offense
\item
  Size of shadow economy
\item
  Feel safe walking alone at night
\end{itemize}

3.3 \emph{Participación cívica}

\begin{itemize}
\tightlist
\item
  Worked in association or organisation
\item
  Signed a petition
\item
  Worn or displayed campaign badge/sticker
\item
  Interest in politics
\item
  Voiced opinion to public official
\item
  Volunteered time to organization
\item
  Voting turnout in elections or referenda
\end{itemize}

\hypertarget{referencias-2}{%
\section{Referencias}\label{referencias-2}}

\begin{enumerate}
\def\labelenumi{\arabic{enumi}.}
\item
  Bertelsmann-Foundation (2013). Social Cohesion Radar. Measuring
  Common Ground. An International Comparison of Social Cohesion.
  Gütersloh, Germany: Bertelsmann-Foundation.

  En este informe los autores presentan los principales resultados del índice de cohesión social para los distintos países en los que ha sido construido. También se incluye una breve conceptualización de lo que comprenden por cohesión social.
\item
  Bertelsmann-Foundation (2013). Social Cohesion Radar. Measuring
  Common Ground. Methods Report. Gütersloh, Germany:
  Bertelsmann-Foundation.

  En este informe se detallan los aspectos metodológicos del índice. Aquí se puede encontrar el detalle de los indicadores utilizados para cada medición y las fuentes de información.
\item
  Delhey, J., \& Dragolov, G. (2015). Happier together. Social cohesion
  and subjective wellbeing in Europe. International Journal of Psychology.

  Este artículo utiliza los datos del índice de cohesión social del radar para comprender cómo esta afecta al bienestar subjetivo para 27 países de la Unión Europea. Los resultados sugieren que europeos en sociedades más cohesivas son más felices y saludables psicológicamente.
\item
  Dragolov, G., Ignácz, Z. S., Lorenz, J., Delhey, J., Boehnke, K., \& Unzicker, K. (2016). Social Cohesion in the Western World. Springer International Publishing. \url{http://link.springer.com/10.1007/978-3-319-32464-7}

  Este libro es una publicación de carácter académico donde se plantean de manera sistemática las bases conceptuales y empíricas de la propuesta del radar de cohesión social.
\end{enumerate}

\hypertarget{civic-engagement-and-social-cohesion-report}{%
\chapter{Civic Engagement and Social Cohesion Report}\label{civic-engagement-and-social-cohesion-report}}

\hypertarget{descripciuxf3n-general-3}{%
\section{Descripción General}\label{descripciuxf3n-general-3}}

La Corporación para Servicios Nacionales y Comunales (CNCS en inglés) destaca la importancia del involucramiento, cohesión social y capital social para los grupos y sociedades, y trabaja en torno a estos tres conceptos. No obstante, dado los objetivos de este reporte el análisis se enfoca en el concepto de cohesión social, sin dejar de considerar la relevancia y relación con los otros dos constructos.

La CNCS encomendó que el Comité en Estadísticas Nacionales creará un panel para identificar mediciones que puedan contribuir a mejorar el entendimiento de los conceptos incorporados en este reporte y su rol en explicar el funcionamiento de la sociedad. El objetivo comisionado a este panel debía considerar los marcos conceptuales, definiciones de conceptos claves, factibilidad y especificaciones de indicadores relevantes, y la relación de estos indicadores y la tendencias sociales. En este sentido, el panel evaluó los méritos de las encuestas, los registros administrativos y datos no gubernamentales. Uno de los estudios sobre capital social referentes para el panel fue el Civic Engagement and Volunteer Supplements of Current Population Survey (CPS), realizado por el U.S. Census Bureau.

La importancia de la recolección de datos y medición de los conceptos de esta propuesta lleva a considerar resultados en dominios específicos como la salud, el crimen, la educación, el empleo y la efectividad del gobierno; el valor de la información descriptiva como insumos para comprender a la sociedad; y su valor político y académico. La relevancia de medir el capital social, involucramiento cívico y cohesión social está también justificada en este caso desde las políticas públicas.

El proyecto fue auspiciado por la CNCS con recursos adicionales de la National Conference on Citizenship (NCoC) y el U.S Office of Management and Budget. Nathan Dietz (CNCS), Christopher Spera (CNCS), John Bridgeland (NCoC), David Smith (NCoC) y Brian Harris-Kojetin (U.S. Office of Management and Budget) definieron para el panel los objetivos guía del estudio.

Un aspecto importante es que el estudio incorpora también prioridades de investigación, desarrollo e implementación a partir de las conclusiones del informe. Sin embargo, el foco principal del estudio es el rol del sistema estadístico federal para mejorar la medición de capital social a través de sus encuestas (\protect\hyperlink{ref-prewitt_Civic_2014}{Prewitt et al., 2014}).

\hypertarget{concepto-de-cohesiuxf3n-social-3}{%
\section{Concepto de Cohesión Social}\label{concepto-de-cohesiuxf3n-social-3}}

El punto de partida del reporte es que el concepto de cohesión social se diferencia de los conceptos de involucramiento cívico y capital social. La cohesión social refiere a la extensión en la cual los grupos y comunidades cooperan, se comunican para mejorar el entendimiento, participan en actividades y organizaciones, y contribuyen para responder a los cambios. Nuevamente estamos frente a un concepto que se sitúa a nivel de grupo y puede corresponder tanto a medios y fines, y no solo a estos últimos. La cohesión social, a diferencia del involucramiento cívico, refiere a las condiciones que hacen posible la acción o son consecuencia de ella. Además, los autores señalan que una sociedad sin cohesión social debería ser una con un gran desorden social y conflicto, valores morales no compartidos por las personas, extrema inequidad, bajos niveles de interacción social entre y en las comunidades, además de bajos niveles de pertenencia.

De forma similar a las demás iniciativas examinadas, el informe destaca la dificultad de medir cohesión social, producto del número importante de dimensiones y la complejidad asociada a cada una de ellas. Por ejemplo, el sentido compartido de moralidad, valores y propósitos comunes; niveles de orden social; extensión de la solidaridad creada por la equidad de ingresos y riqueza; interacción social con y a través de comunidades; y sentido de pertenencia.

La escala espacial es una dimensión esencial del concepto de cohesión social. Barrios, estados, u otros grupos podrían estar en conflicto con otros mientras paralelamente demuestran una fuerte cohesión social interna. En este sentido, es necesario especificar el nivel en el que se está trabajando, cómo se forma la cohesión social y la función que está ocupando en los distintos niveles de la familia a los países y todos los niveles entre ellos.

Un elemento importante a considerar es que los autores del reporte señalan que las actitudes y acciones de los individuos pueden unirlos o separarlos en torno a ellas, por eso es que se deben considerar los clivajes entre grupos opuestos que pueden ser cohesivos en torno a su propia
posición.

\hypertarget{operacionalizaciuxf3n-3}{%
\section{Operacionalización}\label{operacionalizaciuxf3n-3}}

La cohesión se plantea como una subdimensión de la medición propuesta de capital social. En este caso se posiciona como un constructo diferenciado a otros que suelen identificarse como parte de él. Las otras dimensiones son integración social, equidad social, polarización política, confianza en las instituciones, confianza interpersonal, información, involucramiento político y no político. Sin embargo, dado los objetivos de este informe se reporta a continuación la operacionalización de la dimensión cohesión.

Si bien, los autores no llegan al nivel de plantear indicadores si proponen variables y sus características (unidad de análisis, naturaleza del fenómeno y forma de recolección). Para el caso de la cohesión son 8 las variables propuestas, todas posibles de recolectar mediante
encuestas:

\begin{itemize}
\item
  Frecuencia de interacción con familia/amigos - nivel individual
\item
  Ayuda de familia o amigos (redes de apoyo) - nivel individual
\item
  Frecuencia de sentimiento de soledad - nivel individual
\item
  Participación en grupos de chat online - nivel individual
\item
  Vínculos intergrupales - nivel grupal
\item
  Vínculos intragrupales: la unidad de análisis relevante en este caso
  es el grupo, es un fenómeno conductual y que involucra
  características del entorno social - nivel grupal
\item
  Presencia de redes de apoyo - nivel individual
\end{itemize}

\hypertarget{referencias-3}{%
\section{Referencias}\label{referencias-3}}

\begin{enumerate}
\def\labelenumi{\arabic{enumi}.}
\item
  Prewitt, K., Mackie, C. D., \& Habermann, H. (Eds.). (2014). Civic
  Engagement and Social Cohesion:: Measuring Dimensions of Social
  Capital to Inform Policy. Washington, DC.: National Academies
  Press.

  Esta publicación contiene la propuesta general de los autores. Aquí se detalla la revisión de las conceptualizaciones presentes en el informe, entre ellas la cohesión social.
\end{enumerate}

\hypertarget{ecosocial}{%
\chapter{ECOsociAL}\label{ecosocial}}

\hypertarget{descripciuxf3n-general-4}{%
\section{Descripción General}\label{descripciuxf3n-general-4}}

La encuesta ECosociAL-2007 fue realizada con el propósito de medir y documentar el estado de la cohesión social y sus distintas dimensiones en siete países de América Latina: México, Guatemala, Colombia, Brasil, Perú, Argentina y Chile.

El estudio es parte del proyecto ``Una Nueva Agenda para la Cohesión Social en América Latina,'' realizado por la Corporación de Estudios para Latinoamérica (CIEPLAN) de Chile, y el Instituto Fernando Henrique Cardoso (IFHC) de Brasil. ECosociAL-2007 fue financiado por la Comisión Europea, bajo la coordinación del PNUD. El proyecto contó, además, con la participación del Instituto de Sociología de la Pontificia Universidad Católica de Chile y del Helen Kellog Institute for Internacional Studies de la Universidad de Notre Dame, Estados Unidos. El Instituto de Sociología de la Pontificia Universidad Católica de Chile estuvo a cargo de su ejecución y empleó los servicios de instituciones especializadas en cada país donde se aplicó la encuesta.

El cuestionario administrado es resultado de un trabajo conceptual y empírico que involucró a los distintos equipos que participaron en el proyecto entre septiembre del 2006 y febrero del 2007. Esto comenzó con la definición conceptual de cohesión social que se detalla en la siguiente sección.

En cuanto al alcance del estudio, la población objetivo fueron los habitantes de 18 años o más, de ambos sexos, con nacionalidad del país, pertenecientes a todos los niveles socioeconómicos de las principales ciudades incluidas en la investigación en los siete países latinoamericanos.

El tamaño muestral, en base al registro censal de manzanas y cuadras, fue diferente para cada uno de los países del estudio. En Guatemala fue de 1.200 casos; en Argentina, Chile, Colombia y Perú fue de 1.400 casos, en México de 1.500 y finalmente, en Brasil de 1.700 casos. La muestra fue probabilística multietápica hasta la selección de los hogares en donde se seleccionó a los respondentes por cuotas.

Previamente al comienzo del trabajo de campo, el equipo coordinador envió a las instituciones encargadas de la aplicación del instrumento en los distintos países, el detalle del diseño muestral. Una vez consensuado el diseño, cada institución procedió a ejecutar el trabajo de campo.

\hypertarget{concepto-de-cohesiuxf3n-social-4}{%
\section{Concepto de Cohesión Social}\label{concepto-de-cohesiuxf3n-social-4}}

El concepto con el que trabaja ECOsociAL se nutre tanto de las teorías de la sociedad civil como de las teorías de la equidad, tomando una postura ecléctica similar a la de los otras experiencias. En este sentido, la teoría de la sociedad civil sería heredera de la visión tocquevilliana predominante en Estados Unidos y la segunda de la tradición europea y la desestabilización de los Estados de Bienestar. (\protect\hyperlink{ref-somma2015paradojas}{Somma \& Valenzuela, 2015}). La teoría de la sociedad civil hace referencia a la disposición de los sujetos a la cooperación y el compromiso cívico. Los autores señalan que en esta tradición la cohesión social se asocia a la capacidad de producir con confianza social, promover la asociatividad y sancionar a los free-riders. Una sociedad o grupo cohesionado mantendrá redes de cooperación entre extraños y promoverá acciones de los individuos coherentes y con respeto por el bien común. En este sentido, los estudios que utilizan las teorías de la sociedad civil en cohesión social cuantifican la confianza interpersonal, la participación política, la asociatividad y la consistencia entre redes vecinales y de amistad.

Las relaciones de confianza y cooperación tradicionales como la familia, no están consideradas en la sociedad civil, que deja de ser la extensión de la familia para pasar a ser una realidad emergente de relación con otros extraños y diferentes. Incluso, la teoría plantea que las relaciones familiares podrían ser capital social negativo al estar basadas en vínculos con conocidos e iguales. Permanecer entre iguales tiene a la discriminación como su expresión más negativa y, sobre todo, la desintegración producto de la violencia y la criminalidad. Sin embargo, en las teorías de la sociedad civil no es la discriminación el principal elemento de erosión de la cohesión social, sino que el temor que impide la construcción de redes de confianza. (\protect\hyperlink{ref-valenzuela_Vinculos_2008}{Valenzuela et al., 2008})

La segunda tradición considera a la equidad como fuente de la cohesión social en una perspectiva que refiere al fundamento de la estructura social. La equidad se entiende como la capacidad de distribuir equitativamente el poder y el bienestar entre los individuos a través de mecanismos necesarios para lograrlo. En un escenario sin equidad, lo que prima es el conflicto del cuál el antagonismo de clases es el más representativo. Así, la falta de cohesión social se traduciría en polarización entre grupos sociales. Esta conceptualización tiene un lado positivo y un lado negativo. El lado positivo es que la cohesión se consigue mediante arreglos que permitan la distribución equitativa de los bienes y, el lado negativo, es que muchas veces la solución a problemas de cohesión social es el autoritarismo.

ECOsociAL no entrega una definición propia de cohesión social, sino que señala que la medición de la cohesión social en América Latina estará basada en estas dos tradiciones.

\hypertarget{operacionalizaciuxf3n-4}{%
\section{Operacionalización}\label{operacionalizaciuxf3n-4}}

El cuestionario puso énfasis en temáticas como: movilidad social; distribución de oportunidades; legitimación de diferencias socioeconómicas; polarización socio-económica, religiosa y política; confianza social e institucional; exclusión; temor e inseguridad; segregación urbana; solidaridad familiar; legitimación de la violencia política; lealtad democrática; fragmentación étnica y racial; fragmentación territorial y adhesión a Estado nacional.

Como se ha señalado, ECOsociAL trasciende la visión institucional y la inclusión de los grupos excluidos ampliamente difundida en Europa y en experiencias presentes en este informe como la Australiana. Dada las particularidades del continente es que se vuelve necesaria la inclusión del monitoreo del estado de salud de las relaciones de las comunidades y su sustrato cultural, incluyendo la adhesión a la nación.

\hypertarget{la-calidad-de-la-convivencia-social}{%
\subsection{\texorpdfstring{\emph{La calidad de la convivencia social}}{La calidad de la convivencia social}}\label{la-calidad-de-la-convivencia-social}}

\begin{itemize}
\item
  \textbf{Confianza social:} la medición de confianza se realizó a través de dos indicadores: la frase ``se puede confiar en la mayoría de las personas o hay que tener cuidado con ellas'' y la frase ``la mayoría de la gente actúa correctamente con uno o la mayoría trata de aprovecharse.''
\item
  \textbf{Solidaridad familiar:} Para la medición de solidaridad familiar se incluyen tres medidas de apego familiar, ``las personas deben permanecer en contacto con su familia más cercana aún cuando no tengan mucho en común,'' ``las personas deben permanecer en contacto con su familia más lejana como tíos, sobrinos o primos aún cuando no tengan mucho en común,'' y ``en general lo paso mejor con mis amigos que con mi familia.'' A su vez, la solidaridad intergeneracional se mide con tres indicadores: ``cuando los hijos se van de la casa, no deberían esperar que los padres los sigan ayudando económicamente,'' ``cuando los padres envejecen, los hijos deberían hacerse cargo de ellos económicamente'' y ``preferiría que mis hijos solteros se quedaran en casa, aun cuando tengan la capacidad de valerse por sí mismos.''
\item
  \textbf{Relaciones de amistad, vecindad y asociatividad:} la asociatividad se midió a través de las declaraciones de participación en las principales organizaciones sociales. Para las relaciones de amistad y vecindad se pregunta sobre el número de amigos cercanos y de vecinos que se conocen por su nombre. Adicionalmente, para evaluar la calidad de los contactos se pregunta sobre la frecuencia de contactos declarados con familiares, amigos y vecinos.
\item
  \textbf{Tolerancia y discriminación:} ECOsociAL incluye la medición del sentimiento de exclusión a partir de tres indicadores sobre la calidad de la integración en la comunidad próxima: ``en general lo que yo piense no le importa mucho a nadie,'' ``siempre me dejan al margen de las cosas que ocurren a mi alrededor'' y ``siento que la gente que me rodea haría poco para ayudarme si me pasara algo.'' La segregación residencial ha sido estimada mediante la clasificación propia y de los vecinos en general en la escala de estratificación de diez puntos que se ha utilizado en el análisis de la movilidad social. En la discriminación se incluye la frecuencia con que ha sido discriminado por distintas razones: el color de piel, raza o etnia, la religión, la condición de pobreza o preferencia política. Para medir tolerancia se ha preguntado sobre situaciones particulares que comprometen a los hijos como casarse con alguien de una clase social más baja, tener un amigo/a homosexual o casarse con alguien que no tiene religión. Algo similar se realiza para la tolerancia en las relaciones vecinales en donde las situaciones específicas son tener vecinos de una clase social más baja, o tener como vecinos a trabajadores inmigrantes o personas de otra raza. Otro de los aspectos considerados en cuando a la apertura social es la homogamia en tres dimensiones diferentes que son etnia, religión y educación que a su vez es un indicador de apertura social. Se incorpora también la medición de la polarización religiosa a través de la identificación y hostilidad hacia distintas religiones. Lo mismo para la polarización étnica.
\item
  \textbf{Calidad de la vida de barrio, temor y victimización:} la calidad de vida en el barrio se ha medido con tres indicadores para confeccionar un índice de desorganización social que son vandalismo o ataques intencionales a la propiedad privada, robos y asaltos, y balaceras, riñas o violencia callejera; para victimización se consideran los reportes de victimización anual para robo en la casa y en la calle e intimidación con arma de fuego y violencia, cualquiera sea su origen, sea directa o indirecta (``alguien que vive en su casa''); para el temor se utilizan declaraciones de temor o inseguridad para cuando se está sólo en la casa de día o de noche o fuera de la casa, caminando por el barrio o en el centro de la ciudad al anochecer. Dentro de la misma desorganización social se pregunta sobre si se justifica o no poseer un arma de fuego en la casa para defenderse y si se está o no en posesión de una de ella.
\end{itemize}

\hypertarget{la-calidad-de-la-convivencia-poluxedtica}{%
\subsection{\texorpdfstring{\emph{La calidad de la convivencia política}}{La calidad de la convivencia política}}\label{la-calidad-de-la-convivencia-poluxedtica}}

\begin{itemize}
\item
  \textbf{Democracia y derechos:} la calidad de la integración institucional se mide mediante tres frases que son ``a la gente que dirige el país no le importa lo que le pase a personas como uno,'' ``las autoridades no harían nada si hubiera un problema grave en mi barrio o vecindario'' y ``la mayor parte de las personas con poder sólo tratan de aprovecharse de personas como yo.'' La lealtad a la democracia se mide con el grado de acuerdo con las frases ``es mejor la democracia a cualquier otra forma de gobierno'' (que incluye como anverso la preferencia por ``un gobierno de autoridad fuerte en manos de una persona'' y ``da lo mismo una u otra forma de gobierno'') y ``los derechos de las personas se deben respetar en toda circunstancia'' (que tiene como anverso ``los criminales no deben tener los mismos derechos que las personas honestas'').
\item
  \textbf{Confianza en instituciones:} ECosociAL-2007 mide los niveles de confianza declarada en el gobierno, el congreso o parlamento y los alcaldes, ediles o intendentes según sea el caso de cada país. Adicionalmente se incluye la confianza en el presidente, diputados y alcaldes.
\item
  \textbf{Violencia y riesgo político:} la experiencia de vida democrática se mide con el riesgo político observado en el riesgo asociado a los eventos de ``decir lo que se piensa de la política y de los políticos,'' ``participar en partidos políticos de oposición,'' ``participar en manifestaciones contra la autoridad,'' ``ser detenido o maltratado por la policía sin razón aparente,'' ``que la autoridad o policía registre la casa sin orden judicial'' y que ``algún policía, juez o autoridad de gobierno exija un pago, coima o mordida por algo.'' En cuanto a la legitimación de la violencia se preguntó si es justificable que se promueva o defienda determinadas causas usando la fuerza o la violencia. Las causas específicas propuestas son: ``las minorías indígenas que reclaman sus tierras ancestrales'' (violencia étnica); la ``defensa del medio ambiente'' (violencia medioambiental); ``los pobres que reclaman mejores condiciones de vida'' (violencia social); ``cuando se procura hacer cambios revolucionarios en la sociedad'' (violencia revolucionaria); y cuando se trata de ``oponerse a una dictadura'' (violencia democrática).
\item
  \textbf{Distancia con autoridades:} se ha medido la polarización política a través de la identificación y hostilidad hacia el gobierno.
\item
  \textbf{Adhesión a la nación:} el vínculo con la nación es medido a través de cuatro indicadores de nacionalismo, ``tomando todo lo bueno y lo malo, me siento orgulloso de la historia de mi país'' (nacionalismo histórico), ``mi país debería defender sus intereses como nación aun cuando ello conduzca a conflictos con otros países'' (nacionalismo geopolítico) ``mi país debería limitar la importación de productos extranjeros para proteger su economía nacional'' (nacionalismo económico) y ``la televisión de mi país debería dar preferencias a películas y programas nacionales'' (nacionalismo cultural). La identificación se mide respecto a la nación, ciudad/región o etnia, y a su vez se compara la importancia dada a la nación versus las otras dos (más importante la nación, menos importante la nación o igual de importantes).
\end{itemize}

\hypertarget{percepciuxf3n-de-oportunidades-y-movilidad-social}{%
\subsection{\texorpdfstring{\emph{Percepción de oportunidades y movilidad social}}{Percepción de oportunidades y movilidad social}}\label{percepciuxf3n-de-oportunidades-y-movilidad-social}}

\begin{itemize}
\item
  \textbf{Percepción de oportunidades y movilidad social:} la percepción de oportunidades es medida a través de seis indicadores dos de ellos corresponden a percepción de oportunidades educativas (probabilidad de terminar la enseñanza secundaria y de ingresar a la universidad), y los otros cuatro a oportunidades de bienestar (salir de la pobreza, establecerse independientemente, adquirir una vivienda propia y ascender laboralmente cuando se es una mujer).
\item
  \textbf{Identificación socioeconómica:} se ha medido a través de la identificación y hostilidad con clase baja, media y alta.
\item
  \textbf{Oportunidades y vulnerabilidad:} se pregunta sobre las oportunidades que tienen distintos perfiles de personas en el país. Son seis los indicadores incluidos: ``oportunidades de un joven común y corriente de terminar la enseñanza media,'' ``un joven inteligente pero sin recursos de ingresar a la universidad,'' ``cualquier persona de iniciar su propio negocio y establecerse independientemente,'' ``una mujer de alcanzar una buena posición en su trabajo,'' ``cualquier trabajador de adquirir su vivienda en un tiempo razonable'' y ``un pobre de salir de la pobreza.'' Por otro lado, se pregunta sobre la probabilidad de que le ocurran al encuestado o algún miembro de su familia dos eventos específicos: perder empleo y enfermedad que desestabilice el presupuesto familiar.
\item
  \textbf{Movilidad social:} la movilidad educativa se mide a través de la comparación que realiza el encuestado con el nivel educacional de los padres. Para medir la movilidad social los encuestados realizan una clasificación (en una escala de 10 puntos) diseñada para obtener percepciones de la movilidad intrageneracional experimentada (comparación entre autoposicionamiento actual y hace diez años) y movilidad intergeneracional experimentada (comparación con posicionamiento de los padres hace 15 años y autoposicionamiento actual). Asimismo, esto es medido en tanto a expectativa de movilidad social o como esta se proyecta en el tiempo de forma intrageneracional (comparación entre autoposicionamiento actual e intergeneracional (comparación con posicionamiento que proyecta para los hijos cuando tengan la edad actual de quien responde).
\item
  \textbf{Legitimación de diferencias socio-económicas:} se evaluaron por las razones de la riqueza y de la pobreza en dos pares de frases. El primer par apunta a razones adscriptivas (dinero heredado, influencia y contactos en el caso de la riqueza y pobreza heredada y discriminación social en el caso de la pobreza); el segundo par reúne razones adquisitivas o de logro (iniciativa y trabajo duro en el caso de la riqueza y flojera, falta de iniciativa, vicios y alcoholismo en el caso de la pobreza).
\item
  \textbf{Desigualdad:} se incluyen dos pares de oraciones sobre cuál es la mejor sociedad. El primer par es ``la que recompensa el esfuerzo individual'' y ``la que produce mayor igualdad social.'' El segundo par de frases es ``la que permite progresar a cada individuo, aunque crea desigualdad'' y ``la más igualitaria, aunque frene a los más capaces.'' En esta dimensión se incluye el rol del Estado frente a la desigualdad con dos pares de oraciones al igual que el caso anterior: el primer par es ``subir impuestos y aumentar gasto social'' y ``reducir impuestos aunque baje gasto social''; y el segundo par es ``Estado debe darle oportunidades a cada uno'' y ``búsqueda de oportunidades es obligación individual.''
\item
  \textbf{Felicidad:} se mide con un indicador único de la declaración del encuestado si es que se siente feliz.
\end{itemize}

\hypertarget{referencias-4}{%
\section{Referencias}\label{referencias-4}}

\begin{enumerate}
\def\labelenumi{\arabic{enumi}.}
\item
  Valenzuela, E., Schwartzman, S., Biehl, A., \& Valenzuela, J. S. (2008). Vínculos, creencias e ilusiones. La cohesión social de los latinoamericanos.

  Esta es la principal publicación de la Encuesta ECOsociAL 2007. En él se puede encontrar la conceptualización y la revisión de cada uno de los módulos en los distintos capítulos del libro.\\
\item
  Somma, N. M., \& Valenzuela, E. (2015). Las paradojas de la cohesión social en América Latina. Revista del CLAD Reforma y Democracia, (61), 43-74.

  Los autores utilizan los datos de la encuesta ECOsociAL 2007, en complemento con la encuesta mundial de valores, para estudiar tres paradojas que plantean en el caso de la cohesión social latinoamericana en comparación al llamado mundo desarrollado. Las tres paradojas son las siguientes: primero, ¿cómo puede prevalecer en la región más desigual del mundo, que a su vez es bastante rígida en términos de movilidad ocupacional, la creencia de que las desigualdades socioeconómicas obedecen a factores individuales antes que estructurales, así como un insospechado optimismo respecto a la movilidad ascendente futura de los individuos y sus familias? Segundo, ¿cómo pueden mantenerse en pie las instituciones políticas a pesar de los magros desempeños de los gobiernos y los bajos niveles de participación y confianza que suscitan en la población? Y, por último, ¿cómo es posible que los latinoamericanos no sean más intolerantes hacia los demás que los ciudadanos del norte desarrollado, dados sus bajos niveles de confianza interpersonal y la escasa vida asociativa? Además de documentar empíricamente estas paradojas se presentan algunas hipótesis preliminares orientadas a resolverlas.
\end{enumerate}

\hypertarget{conclusiones}{%
\chapter{Conclusiones}\label{conclusiones}}

Dos aspectos generales en los que coinciden las 5 experiencias analizadas es que la cohesión social es un atributo del colectivo y no de los individuos, distanciándose de las conceptualizaciones iniciales que lo homologaban al capital social. Por otro lado, la cohesión social es un constructo multidimensional y la variabilidad está precisamente en las dimensiones que cada una de las experiencias considera.

Como se puede desprender de la Tabla \ref{tab:concepto}, una de los elementos en común que destaca en las conceptualizaciones de cohesión es el foco puesto en un fin compartido por el grupo humano. Estos fines están estrechamente relaciones ya como metas comunes, prosperidad, bien común o dar respuestas colectivas a los cambios. En este sentido, con un sentido teleológico la cohesión social sería la cualidad de las relaciones sociales que permitiría alcanzar estos objetivos en tanto medio. Esta definición involucra elementos importantes como la legitimidad de las instituciones para guiarnos hacia el bien común o la equidad para asegurar el acceso de cada uno de los miembros del grupo al bien producido colectivamente.

\newpage

\begin{longtable}[]{@{}ll@{}}
\caption{\label{tab:concepto} Definiciones de Cohesión social.}\tabularnewline
\toprule
\begin{minipage}[b]{0.10\columnwidth}\raggedright
Proyecto\strut
\end{minipage} & \begin{minipage}[b]{0.85\columnwidth}\raggedright
Descripción\strut
\end{minipage}\tabularnewline
\midrule
\endfirsthead
\toprule
\begin{minipage}[b]{0.10\columnwidth}\raggedright
Proyecto\strut
\end{minipage} & \begin{minipage}[b]{0.85\columnwidth}\raggedright
Descripción\strut
\end{minipage}\tabularnewline
\midrule
\endhead
\begin{minipage}[t]{0.10\columnwidth}\raggedright
Mapping Social Cohesion\strut
\end{minipage} & \begin{minipage}[t]{0.85\columnwidth}\raggedright
La cohesión social está basada en la disposición de los individuos a cooperar y trabajar en todos los niveles de la sociedad para lograr metas comunes\strut
\end{minipage}\tabularnewline
\begin{minipage}[t]{0.10\columnwidth}\raggedright
SMISC\strut
\end{minipage} & \begin{minipage}[t]{0.85\columnwidth}\raggedright
Disposición que tienen los miembros de una sociedad para cooperar con cada uno de los demás para la sobrevivencia y la prosperidad\strut
\end{minipage}\tabularnewline
\begin{minipage}[t]{0.10\columnwidth}\raggedright
Social Cohesion Radar\strut
\end{minipage} & \begin{minipage}[t]{0.85\columnwidth}\raggedright
Una sociedad cohesionada se caracteriza por relaciones sociales resilientes, conexiones emocionales positivas entre los miembros y la comunidad y una foco pronunciado en el bien común.\strut
\end{minipage}\tabularnewline
\begin{minipage}[t]{0.10\columnwidth}\raggedright
CESCR\strut
\end{minipage} & \begin{minipage}[t]{0.85\columnwidth}\raggedright
Refiere a la extensión en la cual los grupos y comunidades cooperan, se comunican para mejorar el entendimiento, participan en actividades y organizaciones, y cooperar para responder a los cambios\strut
\end{minipage}\tabularnewline
\begin{minipage}[t]{0.10\columnwidth}\raggedright
ECOsociAL\strut
\end{minipage} & \begin{minipage}[t]{0.85\columnwidth}\raggedright
1) Capacidad de producir confianza social, promover la asociatividad y sancionar a los free-riders. 2) Equidad como fuente de cohesión social en una perspectiva que refiere al fundamento de la estructura social.\strut
\end{minipage}\tabularnewline
\bottomrule
\end{longtable}

Otro énfasis importante es el dado a la cooperación como soporte actitudinal a un grupo o sociedad cohesionada. Dado que los sujetos tienen un objetivo en común, son capaces de trabajar unos con otros para lograr estos objetivos. Incluso, en ECOsociAL existe referencia a los free-riders, en donde dados los fines comunes y la cooperación una sociedad cohesiva debe asegurar la sanción a los sujetos que no colaboran en esta tarea colectiva.

Si bien las experiencias destacan que no ha existido consenso en clarificar lo que significa cohesión social, podríamos reconocer cuatro elementos comunes en las definiciones utilizadas:

\begin{itemize}
\tightlist
\item
  Atributo del colectivo
\item
  Multidimensionalidad
\item
  Fin común
\item
  Cooperación
\end{itemize}

Otro aspecto de las experiencias internacionales revisadas es la operacionalización que llevan a cabo de la cohesión social. Este paso lógico hace posible identificar indicadores que están en el centro de la discusión y sobre los cuales existe mayor o menor acuerdo como medidas de cohesividad.

\begin{figure}[H]

{\centering \includegraphics[width=0.75\linewidth]{inputs/images/comun} 

}

\caption{Síntesis de dimensiones}\label{fig:dimensiones}
\end{figure}

En un núcleo central, como observamos en la Figura \ref{fig:dimensiones}, se pueden observar las dimensiones que se comparten en mayor medida para medir la cohesión social en una sociedad por las 5 experiencias analizadas. Si bien las experiencias difieren en los indicadores específicos, todas ellas miden de alguna forma la pertenencia, la inclusión, la diversidad y participación. La pertenencia corresponde a las identidades o valores que comparten los grupos y sociedades que en el Social Cohesion Radar es llamada identificación, adhesión a la nación en ECOsociAL o pertenencia propiamente tal en las experiencias canadiense y australiana. Por otra parte, la inclusión es otra de las dimensiones frecuentemente consideradas y que busca medir la forma en que todos los sujetos son parte de forma equitativa de los beneficios de la sociedad. Aquí ECOsociAL hace referencia a la percepción de oportunidades al igual que el Scalon-Monash Index y el Social Cohesion Radar a la percepción de justicia. Una tercera dimensión hace referencia que si bien hablamos de un colectivo, este no necesariamente es homogéneo y que la heterogeneidad no debe afectar la estabilidad o la cohesividad del grupo. Así, diversos indicadores de multiculturalismo, reconocimiento y aceptación de la diversidad son incluidos de forma central por las experiencia analizadas. El índice australiano toma como un foco central de su monitoreo la condición de los inmigrantes y la aceptación de la población nativa de los actuales niveles de inmigración. Asimismo, la experiencia de discriminación es incluida también por el Scalon-Monash Index. Por otro lado, Social Cohesion Radar y ECOsociAL incorporan adicionalmente la diversidad religiosa y sexual.

La participación cívica igualmente es una dimensión central en la medición de cohesión social para la distintas experiencias analizadas. Esto incluye la participación en organizaciones civiles y participación propiamente política. Social Cohesion Radar incluye votación en elecciones, interés en política y el trabajo en asociaciones. Asimismo, la experiencia australiana lo hace solo con participación política incluyendo además de votación otras formas de participación como contacto con representantes, boycot o firma de peticiones. En cambio, el Mapping Social Cohesion de Canadá incluye la participación electoral y la participacion en asociaciones voluntarias. En los márgenes de los elementos centrales, se encuentran otras mediciones de importancia pero que no son ampliamente abordadas por todas las experiencias revisadas. Es el caso de la confianza interpersonal y la confianza en las instituciones. Asimismo, la felicidad o satisfacción con la vida es considerada por la experiencia canadiense y ECOsociAL. Asimismo, las redes sociales son caracterizadas a través del contacto con amigos y sentimiento de soledad. En cuando a la solidaridad se encuentra la donación de dinero y solidaridad familiar. Estas medidas secundarias aparecen al menos en dos de todas las experiencias incluidas en este reporte.

Finalmente, se representan algunos de los indicadores que son incluidos en al menos uno de los monitoreos considerados. Algunas de ellas son la sensación de temor, el riesgo político o el respeto a las normas sociales.

\hypertarget{bibliografuxeda}{%
\chapter*{Bibliografía}\label{bibliografuxeda}}
\addcontentsline{toc}{chapter}{Bibliografía}

\hypertarget{refs}{}
\begin{CSLReferences}{1}{0}
\leavevmode\hypertarget{ref-colic-peisker_Social_2015}{}%
Colic-Peisker, V., \& Robertson, S. (2015). Social change and community cohesion: An ethnographic study of two {Melbourne} suburbs. \emph{Ethnic and Racial Studies}, \emph{38}(1), 75--91. \url{https://doi.org/10.1080/01419870.2014.939205}

\leavevmode\hypertarget{ref-delhey_Happier_2016}{}%
Delhey, J., \& Dragolov, G. (2016). Happier together. {Social} cohesion and subjective well-being in {Europe}: {HAPPIER TOGETHER}-{COHESION AND SWB}. \emph{International Journal of Psychology}, \emph{51}(3), 163--176. \url{https://doi.org/10.1002/ijop.12149}

\leavevmode\hypertarget{ref-dragolov2013social}{}%
Dragolov, G., Ignácz, Z., Lorenz, J., Delhey, J., \& Boehnke, K. (2013). \emph{Social cohesion radar measuring common ground: {An} international comparison of social cohesion methods report}.

\leavevmode\hypertarget{ref-dragolov_Social_2016}{}%
Dragolov, G., Ignácz, Z. S., Lorenz, J., Delhey, J., Boehnke, K., \& Unzicker, K. (2016). \emph{Social {Cohesion} in the {Western World}}. {Springer International Publishing}.

\leavevmode\hypertarget{ref-Jeannote2003}{}%
Jeannotte, M. (2003). \emph{Social cohesion: {Insights} from canadian research}.

\leavevmode\hypertarget{ref-jenson1998mapping}{}%
Jenson, J. (1998). \emph{Mapping social cohesion: {The} state of {Canadian} research}.

\leavevmode\hypertarget{ref-jenson2010defining}{}%
Jenson, J. (2010). \emph{Defining and measuring social cohesion}. {Commonwealth Secretariat}.

\leavevmode\hypertarget{ref-markus2013mapping}{}%
Markus, A. (2014). \emph{Mapping social cohesion}.

\leavevmode\hypertarget{ref-markus_Attitudinal_2007}{}%
Markus, A. B., \& Arunachalam, D. (2007). Attitudinal divergence in a {Melbourne} region of high immigrant concentration: {A} case study. \emph{People and Place}, \emph{15}(4), 38--48.

\leavevmode\hypertarget{ref-ottone2007cohesion}{}%
Ottone, E., Sojo, A., \& CEPAL, N. (2007). \emph{Cohesi{ó}n social: Inclusi{ó}n y sentido de pertenencia en am{é}rica latina y el caribe}.

\leavevmode\hypertarget{ref-prewitt_Civic_2014}{}%
Prewitt, K., Mackie, C. D., Habermann, H., \& Council, N. (2014). \emph{Civic engagement and social cohesion: {Measuring} dimensions of social capital to inform policy}.

\leavevmode\hypertarget{ref-somma2015paradojas}{}%
Somma, N. M., \& Valenzuela, E. (2015). Las paradojas de la cohesión social en {América Latina}. \emph{Revista Del CLAD Reforma y Democracia}, \emph{61}, 43--74.

\leavevmode\hypertarget{ref-tironibarrios_Redes_2008}{}%
Tironi Barrios, E., \& Foxley Ríoseco, A. (Eds.). (2008). \emph{{Redes, Estado y mercados: soportes de la cohesión social latinoamericana}}. {Uqbar Editores}.

\leavevmode\hypertarget{ref-valenzuela_Vinculos_2008}{}%
Valenzuela, E., Schwartzman, S., Biehl, A., \& Valenzuela, J. S. (2008). \emph{Vínculos, creencias e ilusiones. {La} cohesión social de los latinoamericanos}. {Uqbar Editores}.

\end{CSLReferences}

\end{document}
